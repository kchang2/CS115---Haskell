% --------------------------------------------------------------
% This is all preamble stuff that you don't have to worry about.
% Head down to where it says "Start here"
% --------------------------------------------------------------
 
\documentclass[12pt]{article}

\usepackage{enumitem, graphicx, amsmath, subcaption, wrapfig}
\usepackage[margin=1.11in]{geometry}
\setlist[description]{leftmargin=\parindent,labelindent=\parindent}
\graphicspath{ {Users/kaichang/Desktop/ph20} }
\numberwithin{equation}{subsection}


%%%%%% Making LaTeX differentiation / integration much easier %%%%%%%
%%%%% See http://www.latex-community.org/forum/viewtopic.php?f=46&t=10000#p38694 %%%%%
%%%% or C. Beccari. TUGBoat 18 (1997) No. 1. %%%%
\makeatletter
\providecommand*{\diff}%
        {\@ifnextchar^{\DIfF}{\DIfF^{}}}
\def\DIfF^#1{%
        \mathop{\mathrm{\mathstrut d}}%
                \nolimits^{#1}\gobblespace
}
\def\gobblespace{%
        \futurelet\diffarg\opspace}
\def\opspace{%
        \let\DiffSpace\!%
        \ifx\diffarg(%
                \let\DiffSpace\relax
        \else
                \ifx\diffarg\[%
                        \let\DiffSpace\relax
                \else
                        \ifx\diffarg\{%
                                \let\DiffSpace\relax
                        \fi\fi\fi\DiffSpace}   
%%%%%% end here %%%%%%%

%%%% force indent command of paragraph in section %%%%
\newcommand{\forceindent}{\leavevmode{\parindent=1em\indent}}
%%%% end here %%%%

\begin{document}
 
% --------------------------------------------------------------
%                         Start here
% --------------------------------------------------------------
 
\title{CS 115}%replace X with the appropriate number
\author{Kai Chang\\ %replace with your name
SP 2016} %if necessary, replace with your course title
 
\maketitle


%%%%%%%%%%%%
\section{Week 1}

%%%%%%%%%%%%
\section{Week 2}
\subsection{Monday}
\begin{itemize}
\item Packages in Haskell are known as \textit{Hackages}.
\item \textit{:t} stands for type to identify types for variable form.
\item \textit{pointfree} gets rid of all the arguments, just put in your code and then it does what Mathematica is "FullSimplify".
\item goes with input $\rightarrow$ input $\rightarrow$ etc... $\rightarrow$ output
\item \$ is very low precedence, getting us out of writing parenthesis, allowing us to chain things in a nice way.
\item . allows us to remove the redundancy of variables in instancing and computational lines, where you can simply replace the chained variable as a dot. (ie. apply x = p * x + y * x, to apply =  p . y
\item ghci package.hs over ghci and then loading program :l package.hs
\end{itemize}

\end{document}

\section{Week 4}
 14
down vote
accepted
	

You got it, pretty much. So the rest of the deal is designing the predicate function for your list. Assuming you already had a list called xs and a predicate function p, all you'd have to do is

    filter p xs.

Often, you'll see p defined as an anonymous, or lambda, expression, like so:

    filter (\n -> n `mod` 2 == 0) xs.

It is not necessary, and it might be helpful as a beginner to define named functions.

    isEven n = n `mod` 2 == 0

    evenListNumbers xs = filter isEven xs

    evenListNumbers [1,2,3,4]

Which is this [2,4].

So a predicate function for a given list filter takes a list element and returns a boolean value. If it's true, the element is retained (or added to the resulting list), and if it's false, it is passed over.

%%%%%%
